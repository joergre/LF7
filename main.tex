\documentclass{beamer}
\usepackage{beamerthemesplit} % kam neu dazu
\usepackage[ngerman]{babel}
 \usepackage[utf8x]{inputenc}

\begin{document}
\title{VPN} 
\author{Jörg Reuter}
\date{\today} 

\frame{\titlepage} 

\frame{\frametitle{Inhaltsverzeichnis}\tableofcontents} 

\section{VPN - Grundlagen} 
\frame{\frametitle{Mögliche Verbindungen} 
\begin{itemize}
\item Computer - Netzwerk  (End-to-Site)
\item Netzwerk - Netzwerk (Site-to-Site)
\item Client - Server (End-to-End)
\item Server - Server (Host-to-Host (selten), kann aber auch End-to-End bezeichnet werden)
\end{itemize} 
Achtung: Prüfungsrelevant!
}
\frame{\frametitle{Übersicht}
\begin{itemize}
\item Reines Softwareprodukt
\item Es gibt Hardware die das Ausführen von VPN besonders unterstützen
\item B 4.4 VPN des IT-Grundschutz (BSI) beachten!
\item Verbindung über ein anderes Netzwerk
\item entspricht dem Gefühl nach einem direktem Netzwerkanschluss (abgesehen von der Geschwindigkeit)
\item VPN muss dringend Verschlüsselt werden. VPN an sich sieht keine(!) Verschlüsselung vor.
\item Die VPN-Verbindung wird oft als Tunnel bezeichnet
\end{itemize}
}

\section{Praktikum}
\frame{\frametitle{Paktikum}
Wichtige Termine
\begin{itemize}
\item Einführungsseminar 10.09.2015, 10:00-14:00 Uhr
\item Praktikum Schule 38KW-42KW
\item Begleitveranstaltung 1: 24. September, 19:00-22:00 Uhr
\item Begleitveranstaltung 2: 08. Oktober, 19:00-22:00 Uhr
\item Abschlussveranstaltung: 17. Oktober 16:00-22:00 Uhr
\end{itemize}
 }
 
 \section{Abgabe Portfolio}
 \frame{\frametitle{Abgabe Portfolio}
 \begin{itemize}
 \item Abgabe Portfolio am 15. November 2015
 \item Notenbekanntgabe Februar 2016
 
 \end{itemize}
 }
\end{document}